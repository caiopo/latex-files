\documentclass{article}
\usepackage[utf8]{inputenc}
\pagestyle{empty}

\begin{document}
\begin{center}
\section*{Palestra Ciência, Tecnologia e Sociedade (INE5407)}
\begin{center}
Aluno: Caio Pereira Oliveira
\\Data: 20/11/2015
\end{center}

\begin{center}
Título da Palestra: \\ {\bf A Revolucão do ensino e da pesquisa da matemática na Polônia}
Palestrante: {\bf Newton Carneiro Affonso da Costa}
\end{center}

\subsection*{Perguntas}

\end{center}

\begin{enumerate}

{\bf \item Trace um panorama histórico sobre o ensino da matemática na Polônia. Indique os principais nomes que serviram de expoente citados na palestra.}

{\bf Resposta:} Durante a história da atual Polônia, os atuais poloneses foram durante muito tempo subordinados aos vários impérios europeus, isso fez com que o povo polonês criasse um espírito nacionalista muito forte. Então, no início do século XX, alguns matemáticos, vendo o estado deplorável do ensino matemático na Polônia, resolveram traçar um plano para transformar a Polônia em uma referência mundial em ensino da matemática.

Esse plano envolvia melhorar a capacitação de professores, incumbindo-os, além das aulas, a tarefa de indicar ao governo alunos prodigiosos de suas turmas que demonstrassem talento excepcional para a matemática.

O plano também envolveu especializar o ensino da matemática nas três principais universidades (Varsóvia, Cracóvia e Łódź) em linhas diferentes para obter mais profundidade nas pesquisas.

Além disso, os professores universitários foram divididos em dois grupos: um com foco maior em pesquisa, que podia dar menos aulas e outro que não dava tanto foco à pesquisa, mas que dava mais aulas.

Os principais matemáticos citados pelo palestrante por servirem de expoente foram Wacław Sierpiński, Stanisław Leśniewski, Zygmunt Janiszewski e Stefan Mazurkiewicz.

{\bf \item Na sua opinião qual o aspecto mais importante em termos do ensino da matemática na Polônia abordado pelo palestrante?}

{\bf Resposta:} Que os matemáticos poloneses, mesmo em tempos de guerra, resistiram pra manter o desenvolvimento e ensino da matemática.

{\bf \item Qual o aspecto mais relevante da palestra que você destacaria como tema para uma reflexão mais aprofundada? Justifique a sua resposta.}

{\bf Resposta:} O ensino matemático no Brasil e como melhorá-lo.

{\bf \item Na sua opinião quais os principais aspectos abordados na palestra?}

{\bf Resposta:} O rápido crescimento do ensino da matemática na Polônia.

{\bf \item Qual a sua opinião e impressão sobre o palestrante?}

{\bf Resposta:} Um pesquisador e professor respeitável, com inúmeras publicações interessantes, como a fundação da Lógica Paraconsistente, dois artigos sobre o problema N=NP, entre outros.

\end{enumerate}

\end{document}