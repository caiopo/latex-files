\documentclass{article}
\usepackage[utf8]{inputenc}
\usepackage{amsmath}

\begin{document}
\begin{center}
\section*{Palestra Ciência, Tecnologia e Sociedade (INE5407)}
\begin{center}
Nome: Caio Pereira Oliveira
\\Date: 13/09/2015
\end{center}
Título da Palestra: {\bf Tecnologia no Brasil: Uma Visão Geral}

Palestrante: Hamilton Medeiros Silveira, D.Et

\subsection*{Perguntas}

\end{center}

\begin{enumerate} %starts the numbering

{\bf \item Quais os principais aspectos observados e que você destacaria sobre a atuação profissional dos engenheiros na palestra?}

Resposta: 

{\bf \item Quais os principais aspectos observados e que você destacaria sobre o futuro da tecnologia no Brasil na palestra?}

Resposta: O futuro da tecnologia no Brasil é incerto, por falta de interesse do governo e da população em geral, que apenas consome, e não se interessa em produção científico-tecnológica.

{\bf \item No seu entendimento quais os principais aspectos em geral abordados na palestra?}

Resposta: A falta de preparo do país para produzir ciência e tecnologia.

{\bf \item Qual sua opinião e impressão sobre o palestrante?}

Resposta: Muito do que o palestrante falou se mostrou irrelevante para estudantes de Ciência da Computação, principalmente quando ele citou que o objetivo do curso supracitado era apenas de produzir tecnologia, o que é uma visão simplista, pois o curso, como o próprio nome diz, tem sólidas bases teóricas e científicas. Além disso, o palestrante alegou que 

\end{enumerate}

\end{document}
